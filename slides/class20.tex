% Options for packages loaded elsewhere
\PassOptionsToPackage{unicode}{hyperref}
\PassOptionsToPackage{hyphens}{url}
%
\documentclass[
  ignorenonframetext,
]{beamer}
\usepackage{pgfpages}
\setbeamertemplate{caption}[numbered]
\setbeamertemplate{caption label separator}{: }
\setbeamercolor{caption name}{fg=normal text.fg}
\beamertemplatenavigationsymbolshorizontal
% Prevent slide breaks in the middle of a paragraph
\widowpenalties 1 10000
\raggedbottom
\setbeamertemplate{part page}{
  \centering
  \begin{beamercolorbox}[sep=16pt,center]{part title}
    \usebeamerfont{part title}\insertpart\par
  \end{beamercolorbox}
}
\setbeamertemplate{section page}{
  \centering
  \begin{beamercolorbox}[sep=12pt,center]{part title}
    \usebeamerfont{section title}\insertsection\par
  \end{beamercolorbox}
}
\setbeamertemplate{subsection page}{
  \centering
  \begin{beamercolorbox}[sep=8pt,center]{part title}
    \usebeamerfont{subsection title}\insertsubsection\par
  \end{beamercolorbox}
}
\AtBeginPart{
  \frame{\partpage}
}
\AtBeginSection{
  \ifbibliography
  \else
    \frame{\sectionpage}
  \fi
}
\AtBeginSubsection{
  \frame{\subsectionpage}
}

\usepackage{amsmath,amssymb}
\usepackage{iftex}
\ifPDFTeX
  \usepackage[T1]{fontenc}
  \usepackage[utf8]{inputenc}
  \usepackage{textcomp} % provide euro and other symbols
\else % if luatex or xetex
  \usepackage{unicode-math}
  \defaultfontfeatures{Scale=MatchLowercase}
  \defaultfontfeatures[\rmfamily]{Ligatures=TeX,Scale=1}
\fi
\usepackage{lmodern}
\usetheme[]{default}
\ifPDFTeX\else  
    % xetex/luatex font selection
\fi
% Use upquote if available, for straight quotes in verbatim environments
\IfFileExists{upquote.sty}{\usepackage{upquote}}{}
\IfFileExists{microtype.sty}{% use microtype if available
  \usepackage[]{microtype}
  \UseMicrotypeSet[protrusion]{basicmath} % disable protrusion for tt fonts
}{}
\makeatletter
\@ifundefined{KOMAClassName}{% if non-KOMA class
  \IfFileExists{parskip.sty}{%
    \usepackage{parskip}
  }{% else
    \setlength{\parindent}{0pt}
    \setlength{\parskip}{6pt plus 2pt minus 1pt}}
}{% if KOMA class
  \KOMAoptions{parskip=half}}
\makeatother
\usepackage{xcolor}
\newif\ifbibliography
\setlength{\emergencystretch}{3em} % prevent overfull lines
\setcounter{secnumdepth}{-\maxdimen} % remove section numbering

\usepackage{color}
\usepackage{fancyvrb}
\newcommand{\VerbBar}{|}
\newcommand{\VERB}{\Verb[commandchars=\\\{\}]}
\DefineVerbatimEnvironment{Highlighting}{Verbatim}{commandchars=\\\{\}}
% Add ',fontsize=\small' for more characters per line
\usepackage{framed}
\definecolor{shadecolor}{RGB}{241,243,245}
\newenvironment{Shaded}{\begin{snugshade}}{\end{snugshade}}
\newcommand{\AlertTok}[1]{\textcolor[rgb]{0.68,0.00,0.00}{#1}}
\newcommand{\AnnotationTok}[1]{\textcolor[rgb]{0.37,0.37,0.37}{#1}}
\newcommand{\AttributeTok}[1]{\textcolor[rgb]{0.40,0.45,0.13}{#1}}
\newcommand{\BaseNTok}[1]{\textcolor[rgb]{0.68,0.00,0.00}{#1}}
\newcommand{\BuiltInTok}[1]{\textcolor[rgb]{0.00,0.23,0.31}{#1}}
\newcommand{\CharTok}[1]{\textcolor[rgb]{0.13,0.47,0.30}{#1}}
\newcommand{\CommentTok}[1]{\textcolor[rgb]{0.37,0.37,0.37}{#1}}
\newcommand{\CommentVarTok}[1]{\textcolor[rgb]{0.37,0.37,0.37}{\textit{#1}}}
\newcommand{\ConstantTok}[1]{\textcolor[rgb]{0.56,0.35,0.01}{#1}}
\newcommand{\ControlFlowTok}[1]{\textcolor[rgb]{0.00,0.23,0.31}{\textbf{#1}}}
\newcommand{\DataTypeTok}[1]{\textcolor[rgb]{0.68,0.00,0.00}{#1}}
\newcommand{\DecValTok}[1]{\textcolor[rgb]{0.68,0.00,0.00}{#1}}
\newcommand{\DocumentationTok}[1]{\textcolor[rgb]{0.37,0.37,0.37}{\textit{#1}}}
\newcommand{\ErrorTok}[1]{\textcolor[rgb]{0.68,0.00,0.00}{#1}}
\newcommand{\ExtensionTok}[1]{\textcolor[rgb]{0.00,0.23,0.31}{#1}}
\newcommand{\FloatTok}[1]{\textcolor[rgb]{0.68,0.00,0.00}{#1}}
\newcommand{\FunctionTok}[1]{\textcolor[rgb]{0.28,0.35,0.67}{#1}}
\newcommand{\ImportTok}[1]{\textcolor[rgb]{0.00,0.46,0.62}{#1}}
\newcommand{\InformationTok}[1]{\textcolor[rgb]{0.37,0.37,0.37}{#1}}
\newcommand{\KeywordTok}[1]{\textcolor[rgb]{0.00,0.23,0.31}{\textbf{#1}}}
\newcommand{\NormalTok}[1]{\textcolor[rgb]{0.00,0.23,0.31}{#1}}
\newcommand{\OperatorTok}[1]{\textcolor[rgb]{0.37,0.37,0.37}{#1}}
\newcommand{\OtherTok}[1]{\textcolor[rgb]{0.00,0.23,0.31}{#1}}
\newcommand{\PreprocessorTok}[1]{\textcolor[rgb]{0.68,0.00,0.00}{#1}}
\newcommand{\RegionMarkerTok}[1]{\textcolor[rgb]{0.00,0.23,0.31}{#1}}
\newcommand{\SpecialCharTok}[1]{\textcolor[rgb]{0.37,0.37,0.37}{#1}}
\newcommand{\SpecialStringTok}[1]{\textcolor[rgb]{0.13,0.47,0.30}{#1}}
\newcommand{\StringTok}[1]{\textcolor[rgb]{0.13,0.47,0.30}{#1}}
\newcommand{\VariableTok}[1]{\textcolor[rgb]{0.07,0.07,0.07}{#1}}
\newcommand{\VerbatimStringTok}[1]{\textcolor[rgb]{0.13,0.47,0.30}{#1}}
\newcommand{\WarningTok}[1]{\textcolor[rgb]{0.37,0.37,0.37}{\textit{#1}}}

\providecommand{\tightlist}{%
  \setlength{\itemsep}{0pt}\setlength{\parskip}{0pt}}\usepackage{longtable,booktabs,array}
\usepackage{calc} % for calculating minipage widths
\usepackage{caption}
% Make caption package work with longtable
\makeatletter
\def\fnum@table{\tablename~\thetable}
\makeatother
\usepackage{graphicx}
\makeatletter
\def\maxwidth{\ifdim\Gin@nat@width>\linewidth\linewidth\else\Gin@nat@width\fi}
\def\maxheight{\ifdim\Gin@nat@height>\textheight\textheight\else\Gin@nat@height\fi}
\makeatother
% Scale images if necessary, so that they will not overflow the page
% margins by default, and it is still possible to overwrite the defaults
% using explicit options in \includegraphics[width, height, ...]{}
\setkeys{Gin}{width=\maxwidth,height=\maxheight,keepaspectratio}
% Set default figure placement to htbp
\makeatletter
\def\fps@figure{htbp}
\makeatother

\newcommand{\theHtable}{\thetable}
\usepackage{fontspec}
\usepackage{graphicx}
\usepackage{grffile}
\setkeys{Gin}{width=\textwidth,height=\textheight}
\makeatletter
\@ifpackageloaded{caption}{}{\usepackage{caption}}
\AtBeginDocument{%
\ifdefined\contentsname
  \renewcommand*\contentsname{Table of contents}
\else
  \newcommand\contentsname{Table of contents}
\fi
\ifdefined\listfigurename
  \renewcommand*\listfigurename{List of Figures}
\else
  \newcommand\listfigurename{List of Figures}
\fi
\ifdefined\listtablename
  \renewcommand*\listtablename{List of Tables}
\else
  \newcommand\listtablename{List of Tables}
\fi
\ifdefined\figurename
  \renewcommand*\figurename{Figure}
\else
  \newcommand\figurename{Figure}
\fi
\ifdefined\tablename
  \renewcommand*\tablename{Table}
\else
  \newcommand\tablename{Table}
\fi
}
\@ifpackageloaded{float}{}{\usepackage{float}}
\floatstyle{ruled}
\@ifundefined{c@chapter}{\newfloat{codelisting}{h}{lop}}{\newfloat{codelisting}{h}{lop}[chapter]}
\floatname{codelisting}{Listing}
\newcommand*\listoflistings{\listof{codelisting}{List of Listings}}
\makeatother
\makeatletter
\makeatother
\makeatletter
\@ifpackageloaded{caption}{}{\usepackage{caption}}
\@ifpackageloaded{subcaption}{}{\usepackage{subcaption}}
\makeatother

\ifLuaTeX
  \usepackage{selnolig}  % disable illegal ligatures
\fi
\usepackage{bookmark}

\IfFileExists{xurl.sty}{\usepackage{xurl}}{} % add URL line breaks if available
\urlstyle{same} % disable monospaced font for URLs
\hypersetup{
  pdftitle={Class 20},
  pdfauthor={Sarah E. Grabinski},
  hidelinks,
  pdfcreator={LaTeX via pandoc}}


\title{Class 20}
\subtitle{DATA1220-55, Fall 2024}
\author{Sarah E. Grabinski}
\date{2024-10-18}

\begin{document}
\frame{\titlepage}


\begin{frame}{Hypothesis Testing Framework}
\phantomsection\label{hypothesis-testing-framework}
\begin{itemize}
\item
  \(\mathbf{H_0}\): The ``Null'' Hypothesis

  \begin{itemize}
  \item
    Represents a position of skepticism, \emph{nothing} is happening
    here
  \item
    ``There is \emph{not} an association between process A and B''
  \end{itemize}
\end{itemize}

\pause

\begin{itemize}
\item
  \(\mathbf{H_A}\): The ``Alternative'' Hypothesis

  \begin{itemize}
  \item
    The complement of \(H_0\), \emph{something} is happening here
  \item
    ``There \emph{is} an association between process A and B''
  \end{itemize}
\end{itemize}
\end{frame}

\begin{frame}{Conducting a hypothesis test}
\phantomsection\label{conducting-a-hypothesis-test}
\begin{itemize}
\tightlist
\item
  Begin by \emph{assuming} \(H_0\) is the ``true'' state
\end{itemize}

\pause

\begin{itemize}
\tightlist
\item
  Calculate \emph{the probability that you would see results as extreme
  or more extreme} than what you saw in your study, assuming the
  distribution under \(H_0\)
\end{itemize}

\pause

\begin{itemize}
\tightlist
\item
  The lower the probability, the less likely it is that we would see
  these results if \(H_0\) was the ``true'' state of our population
\end{itemize}

\pause

\begin{itemize}
\tightlist
\item
  If the probability is sufficiently low, we \emph{reject}
  \(\mathbf{H_0}\) and \emph{accept} \(\mathbf{H_A}\)
\end{itemize}
\end{frame}

\begin{frame}{Example: One Proportion}
\phantomsection\label{example-one-proportion}
In a 2024 Ipsos survey of a representative sample of 2,027 Americans,
1,155 respondents (57.0\%) reported that they had a favorable opinion of
Tom Hanks, down from 75\% in 2018. \begin{center}
\includegraphics{class20_files/mediabag/Slide5_247.JPG}
\end{center}
\end{frame}

\begin{frame}{Hypotheses}
\phantomsection\label{hypotheses}
\pause

\begin{itemize}
\tightlist
\item
  \(H_0\): Americans do not have an opinion on Tom Hanks.
\end{itemize}

\pause

\[P(\operatorname{Favorable}) = 0.5\]

\pause

\begin{itemize}
\tightlist
\item
  \(H_A\): Americans do have an opinion on Tom Hanks.
\end{itemize}

\pause

\[P(\operatorname{Favorable}) \ne 0.5\]
\end{frame}

\begin{frame}{95\% Confidence Interval}
\phantomsection\label{confidence-interval}
\pause

\begin{enumerate}
\tightlist
\item
  Calculate the standard error of the measurement for a proportion.
\end{enumerate}

\[
SE=\sqrt{\frac{\hat{p}(1-\hat{p})}{n}}
\]

\pause

\begin{enumerate}
\setcounter{enumi}{1}
\tightlist
\item
  Find \(Z^*\) for \(\alpha = 0.05\).
\end{enumerate}

\[
Z^*=Z_{1-\alpha/2}
\]

\pause

\begin{enumerate}
\setcounter{enumi}{2}
\tightlist
\item
  Construct confidence interval as
  \(\operatorname{point estimate} \pm Z^* \times SE\)
\end{enumerate}
\end{frame}

\begin{frame}[fragile]{Standard Error}
\phantomsection\label{standard-error}
\[
\begin{aligned}
SE&=\sqrt{\frac{\hat{p}(1-\hat{p})}{n}} \\
&= \sqrt{\frac{0.57(1-0.57)}{2027}} \\
&=0.011
\end{aligned}
\]

\pause

\begin{Shaded}
\begin{Highlighting}[]
\FunctionTok{sqrt}\NormalTok{((}\FloatTok{0.57} \SpecialCharTok{*}\NormalTok{ (}\DecValTok{1} \SpecialCharTok{{-}} \FloatTok{0.57}\NormalTok{)) }\SpecialCharTok{/} \DecValTok{2027}\NormalTok{)}
\end{Highlighting}
\end{Shaded}

\begin{verbatim}
[1] 0.01099625
\end{verbatim}
\end{frame}

\begin{frame}[fragile]{Finding \(Z^*\)}
\phantomsection\label{finding-z}
\[
\begin{aligned}
Z^*&=Z_{1-\alpha/2} \\
&=Z_{1-0.025} \\
&=Z_{0.975}
\end{aligned}
\]

\pause

\begin{Shaded}
\begin{Highlighting}[]
\FunctionTok{qnorm}\NormalTok{(}\FloatTok{0.975}\NormalTok{)}
\end{Highlighting}
\end{Shaded}

\begin{verbatim}
[1] 1.959964
\end{verbatim}
\end{frame}

\begin{frame}[fragile]{Calculating the Margin of Error}
\phantomsection\label{calculating-the-margin-of-error}
\[
\begin{aligned}
\operatorname{margin of error} &= Z^* \times SE \\
&= 1.96 \times 0.011 \\
&= 0.022
\end{aligned}
\]

\pause

\begin{Shaded}
\begin{Highlighting}[]
\FunctionTok{qnorm}\NormalTok{(}\FloatTok{0.975}\NormalTok{) }\SpecialCharTok{*} \FunctionTok{sqrt}\NormalTok{((}\FloatTok{0.57} \SpecialCharTok{*}\NormalTok{ (}\DecValTok{1} \SpecialCharTok{{-}} \FloatTok{0.57}\NormalTok{)) }\SpecialCharTok{/} \DecValTok{2027}\NormalTok{)}
\end{Highlighting}
\end{Shaded}

\begin{verbatim}
[1] 0.02155226
\end{verbatim}
\end{frame}

\begin{frame}[fragile]{Finding the boundaries}
\phantomsection\label{finding-the-boundaries}
\begin{Shaded}
\begin{Highlighting}[]
\NormalTok{(}\DecValTok{1155} \SpecialCharTok{/} \DecValTok{2027}\NormalTok{) }\SpecialCharTok{{-}} \FunctionTok{qnorm}\NormalTok{(}\FloatTok{0.975}\NormalTok{) }\SpecialCharTok{*} \FunctionTok{sqrt}\NormalTok{((}\FloatTok{0.57} \SpecialCharTok{*}\NormalTok{ (}\DecValTok{1} \SpecialCharTok{{-}} \FloatTok{0.57}\NormalTok{)) }\SpecialCharTok{/} \DecValTok{2027}\NormalTok{)}
\end{Highlighting}
\end{Shaded}

\begin{verbatim}
[1] 0.5482553
\end{verbatim}

\begin{Shaded}
\begin{Highlighting}[]
\NormalTok{(}\DecValTok{1155} \SpecialCharTok{/} \DecValTok{2027}\NormalTok{) }\SpecialCharTok{+} \FunctionTok{qnorm}\NormalTok{(}\FloatTok{0.975}\NormalTok{) }\SpecialCharTok{*} \FunctionTok{sqrt}\NormalTok{((}\FloatTok{0.57} \SpecialCharTok{*}\NormalTok{ (}\DecValTok{1} \SpecialCharTok{{-}} \FloatTok{0.57}\NormalTok{)) }\SpecialCharTok{/} \DecValTok{2027}\NormalTok{)}
\end{Highlighting}
\end{Shaded}

\begin{verbatim}
[1] 0.5913599
\end{verbatim}
\end{frame}

\begin{frame}{Interpreting the Confidence Interval}
\phantomsection\label{interpreting-the-confidence-interval}
\begin{itemize}
\tightlist
\item
  With 95\% confidence, 54.8\% to 59.1\% of Americans have a favorable
  opinion of Tom Hanks.
\end{itemize}

\pause

\begin{itemize}
\tightlist
\item
  57.0\% of Americans have a favorable opinion of Tom Hanks (95\% CI:
  54.8-59.1\%).
\end{itemize}

\pause

\begin{itemize}
\tightlist
\item
  With 95\% confidence, 57.0\% \(\pm\) 2.2\% have a favorable opinion of
  Tom Hanks.
\end{itemize}
\end{frame}

\begin{frame}{Testing the Hypothesis}
\phantomsection\label{testing-the-hypothesis}
\begin{enumerate}
\tightlist
\item
  Find the sampling distribution \(N(p, SE)\) under the null hypothesis.
\end{enumerate}

\pause

\begin{enumerate}
\setcounter{enumi}{1}
\tightlist
\item
  Calculate the test statistic for the null hypothesis.
\end{enumerate}

\[
Z=\frac{\hat{p}-p}{SE}
\]

\pause

\begin{enumerate}
\setcounter{enumi}{2}
\tightlist
\item
  Find the probability of getting a test statistic as extreme or more
  extreme as this one, assuming the null hypothesis is true.
\end{enumerate}

\pause

\begin{enumerate}
\setcounter{enumi}{3}
\tightlist
\item
  If the p-value is less than \(\alpha\), reject \(H_0\) and accept
  \(H_A\).
\end{enumerate}
\end{frame}

\begin{frame}{Finding the Null Distribution}
\phantomsection\label{finding-the-null-distribution}
To find the null distribution, replace the sample statistic
\(\hat{p}=0.57\) with the population parameter \(p=0.5\).

\[
\begin{aligned}
SE&=\sqrt{\frac{p(1-p)}{n}} \\
&= \sqrt{\frac{0.5(1-0.5)}{2027}} \\
&=0.011
\end{aligned}
\]

\pause

The null hypothesis is that our observed sample statistic
\(\hat{p}=0.57\) comes from the sampling distribution
\(N(\mu=0.5, \sigma=0.011)\).
\end{frame}

\begin{frame}{Calculating the test statistic}
\phantomsection\label{calculating-the-test-statistic}
If the sampling distribution under \(H_0\) is
\(\hat{p}\sim N(0.5, 0.011)\), then the test statistic for
\(\hat{p}=0.57\) is\ldots{}

\[
\begin{aligned}
Z &= \frac{\hat{p} - p}{SE} \\
&= \frac{0.57-0.5}{0.011} \\
&=6.36
\end{aligned}
\]

\pause

\(\hat{p}\) is 6.36 standard errors greater than \(p\) under the null
hypothesis.
\end{frame}

\begin{frame}[fragile]{Get the p-value}
\phantomsection\label{get-the-p-value}
\begin{itemize}
\tightlist
\item
  Use the test statistic \(Z\) to find the two-sided probability
  \(P(Z \ge 6.36 \operatorname{or} Z \le -6.36)\).
\end{itemize}

\begin{Shaded}
\begin{Highlighting}[]
\FunctionTok{pnorm}\NormalTok{(}\SpecialCharTok{{-}}\FloatTok{6.36}\NormalTok{) }\SpecialCharTok{+} \FunctionTok{pnorm}\NormalTok{(}\FloatTok{6.36}\NormalTok{, }\AttributeTok{lower.tail =}\NormalTok{ F)}
\end{Highlighting}
\end{Shaded}

\begin{verbatim}
[1] 2.017537e-10
\end{verbatim}

\pause

\begin{itemize}
\tightlist
\item
  If \(H_0\):\(P(\operatorname{Favorable}) = 0.5\) were true, then the
  probability that we would see a sample proportion as different from
  \(p=0.5\) as \(\hat{p}=0.57\) is very low.
\end{itemize}

\pause

\begin{itemize}
\tightlist
\item
  \(P(\lvert Z \rvert \ge 6.36) < 0.05\), so we \emph{reject} \(H_0\)
  and \emph{accept} \(H_A\).
\end{itemize}
\end{frame}

\begin{frame}{Difference Between 2 Proportions}
\phantomsection\label{difference-between-2-proportions}
\begin{itemize}
\tightlist
\item
  In 2018, Ipsos surveyed 1,005 Americans using the same questions, and
  754 (75.0\%) had a favorable opinion of Tom Hanks.
\end{itemize}

\pause

\begin{itemize}
\tightlist
\item
  Research Question: Has Tom Hanks' favorability dropped between 2018
  and 2024?
\end{itemize}

\pause

\begin{itemize}
\tightlist
\item
  Does \(\hat{p}_{2018}=\hat{p}_{2024}\)? Does
  \(\hat{p}_{2018}-\hat{p}_{2024}=0\)?
\end{itemize}
\end{frame}

\begin{frame}{Sampling Distribution}
\phantomsection\label{sampling-distribution}
\begin{itemize}
\tightlist
\item
  Sample statistic is the difference between 2 sample proportions
  \(\hat{p}_1 - \hat{p}_2\)
\end{itemize}

\pause

\begin{itemize}
\tightlist
\item
  When assumptions met,
  \(\hat{p}_1-\hat{p}_2 \sim N(p_1-p_2, SE_{p_1-p_2})\)
\end{itemize}
\end{frame}

\begin{frame}{Standard Error}
\phantomsection\label{standard-error-1}
The standard error for the difference between 2 proportions requires the
population proportion \(p\) and sample size \(n\) for each group.

\[
SE_{(\hat{p}_1-\hat{p}_2)}=\sqrt{\frac{p_1(1-p_1)}{n_1}+\frac{p_2(1-p_2)}{n_2}}
\]
\end{frame}

\begin{frame}{Assumptions}
\phantomsection\label{assumptions}
\begin{itemize}
\tightlist
\item
  Observations within each sample are independent of each other.
\end{itemize}

\pause

\begin{itemize}
\tightlist
\item
  There are 10+ successes and 10+ failures in each sample.
\end{itemize}

\pause

\begin{itemize}
\tightlist
\item
  Sample 1 is independent of Sample 2.
\end{itemize}
\end{frame}

\begin{frame}{Hypotheses}
\phantomsection\label{hypotheses-1}
\pause

\begin{itemize}
\tightlist
\item
  \(H_0\): The proportion of Americans who view Tom Hanks favorably did
  not change between 2018 and 2024.
\end{itemize}

\pause

\[P(\operatorname{Favorable in 2018}) = P(\operatorname{Favorable in 2024})\]

\pause

\begin{itemize}
\tightlist
\item
  \(H_A\): The proportion of Americans who view Tom Hanks favorably
  changed between 2018 and 2024.
\end{itemize}

\pause

\[P(\operatorname{Favorable in 2018}) \ne P(\operatorname{Favorable in 2024})\]
\end{frame}

\begin{frame}{Hypotheses}
\phantomsection\label{hypotheses-2}
\begin{itemize}
\tightlist
\item
  \(H_0\): The proportion of Americans who view Tom Hanks favorably did
  not change between 2018 and 2024.
\end{itemize}

\[P(\operatorname{Favorable in 2018}) - P(\operatorname{Favorable in 2024})=0\]

\begin{itemize}
\tightlist
\item
  \(H_A\): The proportion of Americans who view Tom Hanks favorably
  changed between 2018 and 2024.
\end{itemize}

\[P(\operatorname{Favorable in 2018}) - P(\operatorname{Favorable in 2024}) \ne 0\]
\end{frame}

\begin{frame}{95\% Confidence Interval}
\phantomsection\label{confidence-interval-1}
\begin{enumerate}
\tightlist
\item
  Find the point estimate for the difference in proportions
  \(\hat{p}_1-\hat{p}_2\).
\end{enumerate}

\pause

\begin{enumerate}
\setcounter{enumi}{1}
\tightlist
\item
  Calculate the standard error for the difference between 2 proportions.
\end{enumerate}

\[
SE_{(\hat{p}_1-\hat{p}_2)}=\sqrt{\frac{p_1(1-p_1)}{n_1}+\frac{p_2(1-p_2)}{n_2}}
\]

\pause

\begin{enumerate}
\setcounter{enumi}{1}
\tightlist
\item
  Find \(Z^*\) for \(\alpha = 0.05\).
\end{enumerate}

\[
Z^*=Z_{1-\alpha/2}
\]

\pause

\begin{enumerate}
\setcounter{enumi}{2}
\tightlist
\item
  Construct confidence interval as
  \(\hat{p}_1 - \hat{p}_2 \pm Z^* \times SE\)
\end{enumerate}
\end{frame}

\begin{frame}{Point Estimate}
\phantomsection\label{point-estimate}
\[
\begin{aligned}
\hat{p}_1-\hat{p}_2&=\frac{754}{1005}-\frac{1155}{2027} \\
&= 0.180
\end{aligned}
\]
\end{frame}

\begin{frame}{Standard Error}
\phantomsection\label{standard-error-2}
When constructing a confidence interval for \(\hat{p}_1-\hat{p}_2\), we
use the sample proportions \(\hat{p}_1\) and \(\hat{p}_2\) as estimates
for the population parameters \(p_1\) and \(p_2\).

\pause

\[
\begin{aligned}
SE_{(\hat{p}_1-\hat{p}_2)}&=\sqrt{\frac{p_1(1-p_1)}{n_1}+\frac{p_2(1-p_2)}{n_2}} \\
&=\sqrt{\frac{0.75(1-0.75)}{1005}+\frac{0.57(1-0.57)}{2027}} \\
&=0.018
\end{aligned}
\]
\end{frame}

\begin{frame}[fragile]{Finding \(Z^*\)}
\phantomsection\label{finding-z-1}
\[
\begin{aligned}
Z^*&=Z_{1-\alpha/2} \\
&=Z_{1-0.025} \\
&=Z_{0.975}
\end{aligned}
\]

\pause

\begin{Shaded}
\begin{Highlighting}[]
\FunctionTok{qnorm}\NormalTok{(}\FloatTok{0.975}\NormalTok{)}
\end{Highlighting}
\end{Shaded}

\begin{verbatim}
[1] 1.959964
\end{verbatim}
\end{frame}

\begin{frame}[fragile]{Calculating the Margin of Error}
\phantomsection\label{calculating-the-margin-of-error-1}
\[
\begin{aligned}
\operatorname{margin of error} &= Z^* \times SE \\
&= 1.96 \times 0.018 \\
&= 0.034
\end{aligned}
\]

\pause

\begin{Shaded}
\begin{Highlighting}[]
\FunctionTok{qnorm}\NormalTok{(}\FloatTok{0.975}\NormalTok{) }\SpecialCharTok{*} \FunctionTok{sqrt}\NormalTok{((}\FloatTok{0.75}\SpecialCharTok{*}\NormalTok{(}\DecValTok{1}\FloatTok{{-}0.75}\NormalTok{))}\SpecialCharTok{/}\DecValTok{1005} \SpecialCharTok{+}\NormalTok{ (}\FloatTok{0.57}\SpecialCharTok{*}\NormalTok{(}\DecValTok{1}\FloatTok{{-}0.57}\NormalTok{))}\SpecialCharTok{/}\DecValTok{2027}\NormalTok{)}
\end{Highlighting}
\end{Shaded}

\begin{verbatim}
[1] 0.03436845
\end{verbatim}
\end{frame}

\begin{frame}[fragile]{Finding the boundaries}
\phantomsection\label{finding-the-boundaries-1}
\begin{Shaded}
\begin{Highlighting}[]
\NormalTok{diff }\OtherTok{\textless{}{-}}\NormalTok{ ((}\DecValTok{754} \SpecialCharTok{/} \DecValTok{1005}\NormalTok{) }\SpecialCharTok{{-}}\NormalTok{ (}\DecValTok{1155} \SpecialCharTok{/} \DecValTok{2027}\NormalTok{))}

\NormalTok{margin }\OtherTok{\textless{}{-}} \FunctionTok{qnorm}\NormalTok{(}\FloatTok{0.975}\NormalTok{) }\SpecialCharTok{*} \FunctionTok{sqrt}\NormalTok{((}\FloatTok{0.75}\SpecialCharTok{*}\NormalTok{(}\DecValTok{1}\FloatTok{{-}0.75}\NormalTok{))}\SpecialCharTok{/}\DecValTok{1005} \SpecialCharTok{+}\NormalTok{ (}\FloatTok{0.57}\SpecialCharTok{*}\NormalTok{(}\DecValTok{1}\FloatTok{{-}0.57}\NormalTok{))}\SpecialCharTok{/}\DecValTok{2027}\NormalTok{)}

\NormalTok{diff }\SpecialCharTok{{-}}\NormalTok{ margin}
\end{Highlighting}
\end{Shaded}

\begin{verbatim}
[1] 0.1460727
\end{verbatim}

\begin{Shaded}
\begin{Highlighting}[]
\NormalTok{diff }\SpecialCharTok{+}\NormalTok{ margin}
\end{Highlighting}
\end{Shaded}

\begin{verbatim}
[1] 0.2148096
\end{verbatim}
\end{frame}

\begin{frame}{Interpreting the Confidence Interval}
\phantomsection\label{interpreting-the-confidence-interval-1}
\begin{itemize}
\tightlist
\item
  With 95\% confidence, 14.6\% to 21.5\% more Americans had a favorable
  opinion of Tom Hanks in 2018 than in 2024.
\end{itemize}

\pause

\begin{itemize}
\tightlist
\item
  18.0\% more Americans had a favorable opinion of Tom Hanks in 2018
  than in 2024 (95\% CI: 14.6-21.5\%).
\end{itemize}

\pause

\begin{itemize}
\tightlist
\item
  With 95\% confidence, 18.0\% \(\pm\) 3.4\% more Americans had a
  favorable opinion of Tom Hanks in 2018 than 2024.
\end{itemize}
\end{frame}

\begin{frame}{Testing the Hypothesis}
\phantomsection\label{testing-the-hypothesis-1}
\begin{enumerate}
\tightlist
\item
  Find the sampling distribution \(N(0, SE_{p_1-p_2})\) under the null
  hypothesis.
\end{enumerate}

\pause

\begin{enumerate}
\setcounter{enumi}{1}
\tightlist
\item
  Calculate the test statistic for the null hypothesis \(p_1=p_2\).
\end{enumerate}

\[
Z=\frac{\hat{p}_1-\hat{p}_2}{SE_{p_1=p_2}}
\]

\pause

\begin{enumerate}
\setcounter{enumi}{2}
\tightlist
\item
  Find the probability of getting a test statistic as extreme or more
  extreme as this one, assuming the null hypothesis is true.
\end{enumerate}

\pause

\begin{enumerate}
\setcounter{enumi}{3}
\tightlist
\item
  If the p-value is less than \(\alpha\), reject \(H_0\) and accept
  \(H_A\).
\end{enumerate}
\end{frame}

\begin{frame}{Pooled Population Proportion}
\phantomsection\label{pooled-population-proportion}
When the null hypothesis is that \(p_1=p_2\) or \(p_1-p_2=0\), we use
the pooled population parameter \(\hat{p}\) to calculate the standard
error.

\[
\begin{aligned}
\hat{p}_{\operatorname{pooled}}&=\frac{\operatorname{count}_1 + \operatorname{count}_2}{n_1+n_2} \\
&=\frac{754 + 1155}{1005 + 2027} \\
&=0.630
\end{aligned}
\]

\pause

When \(H_0\): \(p_1-p_2 \ne 0\), then you use \(\hat{p}_1\) and
\(\hat{p}_2\) when calculating the standard error for the hypothesis
test.
\end{frame}

\begin{frame}{Finding the Null Distribution}
\phantomsection\label{finding-the-null-distribution-1}
To find the null distribution for \(p_1=p_2\), replace the sample
statistics \(\hat{p}_1\) and \(\hat{p}_2\) with the pooled population
proportion \(\hat{p}\).

\[
\begin{aligned}
SE_{(p_1=p_2)}&=\sqrt{\frac{\hat{p}(1-\hat{p})}{n_1}+\frac{\hat{p}(1-\hat{p})}{n_2}} \\
&=\sqrt{\frac{0.630(1-0.630)}{1005}+\frac{0.630(1-0.630)}{2027}} \\
&=0.019
\end{aligned}
\]

\pause

The null hypothesis is that our observed sample statistic
\(\hat{p}_1-\hat{p}_2\) comes from the sampling distribution
\(N(\mu=0, \sigma=0.019)\).
\end{frame}

\begin{frame}{Calculating the test statistic}
\phantomsection\label{calculating-the-test-statistic-1}
If the sampling distribution under \(H_0\) is
\(\hat{p}_1-\hat{p}_2 \sim N(0, 0.019)\), then the test statistic for
\(\hat{p}_1-\hat{p}_2=\) is\ldots{}

\[
\begin{aligned}
Z &= \frac{\hat{p} - p}{SE} \\
&= \frac{0.750-0.570}{0.019} \\
&=9.47
\end{aligned}
\]

\pause

\(\hat{p}_1-\hat{p}_2\) is 9.47 standard errors greater than
\(\hat{p}_1-\hat{p}_2=0\) under the null hypothesis.
\end{frame}

\begin{frame}[fragile]{Get the p-value}
\phantomsection\label{get-the-p-value-1}
\begin{itemize}
\tightlist
\item
  Use the test statistic \(Z\) to find the two-sided probability
  \(P(Z \ge 9.47 \operatorname{or} Z \le -9.47)\).
\end{itemize}

\begin{Shaded}
\begin{Highlighting}[]
\FunctionTok{pnorm}\NormalTok{(}\SpecialCharTok{{-}}\FloatTok{9.47}\NormalTok{) }\SpecialCharTok{+} \FunctionTok{pnorm}\NormalTok{(}\FloatTok{9.47}\NormalTok{, }\AttributeTok{lower.tail =}\NormalTok{ F)}
\end{Highlighting}
\end{Shaded}

\begin{verbatim}
[1] 2.798437e-21
\end{verbatim}

\pause

\begin{itemize}
\tightlist
\item
  If \(H_0\):\(p_{2018}=p_{2024}\) were true, then the probability that
  we would see a difference as large as 18.0\% is very small.
\end{itemize}

\pause

\begin{itemize}
\tightlist
\item
  \(P(\lvert Z \rvert \ge 9.47) < 0.05\), so we \emph{reject} \(H_0\)
  and \emph{accept} \(H_A\).
\end{itemize}
\end{frame}




\end{document}
