% Options for packages loaded elsewhere
\PassOptionsToPackage{unicode}{hyperref}
\PassOptionsToPackage{hyphens}{url}
%
\documentclass[
  ignorenonframetext,
]{beamer}
\usepackage{pgfpages}
\setbeamertemplate{caption}[numbered]
\setbeamertemplate{caption label separator}{: }
\setbeamercolor{caption name}{fg=normal text.fg}
\beamertemplatenavigationsymbolshorizontal
% Prevent slide breaks in the middle of a paragraph
\widowpenalties 1 10000
\raggedbottom
\setbeamertemplate{part page}{
  \centering
  \begin{beamercolorbox}[sep=16pt,center]{part title}
    \usebeamerfont{part title}\insertpart\par
  \end{beamercolorbox}
}
\setbeamertemplate{section page}{
  \centering
  \begin{beamercolorbox}[sep=12pt,center]{part title}
    \usebeamerfont{section title}\insertsection\par
  \end{beamercolorbox}
}
\setbeamertemplate{subsection page}{
  \centering
  \begin{beamercolorbox}[sep=8pt,center]{part title}
    \usebeamerfont{subsection title}\insertsubsection\par
  \end{beamercolorbox}
}
\AtBeginPart{
  \frame{\partpage}
}
\AtBeginSection{
  \ifbibliography
  \else
    \frame{\sectionpage}
  \fi
}
\AtBeginSubsection{
  \frame{\subsectionpage}
}

\usepackage{amsmath,amssymb}
\usepackage{iftex}
\ifPDFTeX
  \usepackage[T1]{fontenc}
  \usepackage[utf8]{inputenc}
  \usepackage{textcomp} % provide euro and other symbols
\else % if luatex or xetex
  \usepackage{unicode-math}
  \defaultfontfeatures{Scale=MatchLowercase}
  \defaultfontfeatures[\rmfamily]{Ligatures=TeX,Scale=1}
\fi
\usepackage{lmodern}
\usetheme[]{default}
\ifPDFTeX\else  
    % xetex/luatex font selection
\fi
% Use upquote if available, for straight quotes in verbatim environments
\IfFileExists{upquote.sty}{\usepackage{upquote}}{}
\IfFileExists{microtype.sty}{% use microtype if available
  \usepackage[]{microtype}
  \UseMicrotypeSet[protrusion]{basicmath} % disable protrusion for tt fonts
}{}
\makeatletter
\@ifundefined{KOMAClassName}{% if non-KOMA class
  \IfFileExists{parskip.sty}{%
    \usepackage{parskip}
  }{% else
    \setlength{\parindent}{0pt}
    \setlength{\parskip}{6pt plus 2pt minus 1pt}}
}{% if KOMA class
  \KOMAoptions{parskip=half}}
\makeatother
\usepackage{xcolor}
\newif\ifbibliography
\setlength{\emergencystretch}{3em} % prevent overfull lines
\setcounter{secnumdepth}{-\maxdimen} % remove section numbering


\providecommand{\tightlist}{%
  \setlength{\itemsep}{0pt}\setlength{\parskip}{0pt}}\usepackage{longtable,booktabs,array}
\usepackage{calc} % for calculating minipage widths
\usepackage{caption}
% Make caption package work with longtable
\makeatletter
\def\fnum@table{\tablename~\thetable}
\makeatother
\usepackage{graphicx}
\makeatletter
\def\maxwidth{\ifdim\Gin@nat@width>\linewidth\linewidth\else\Gin@nat@width\fi}
\def\maxheight{\ifdim\Gin@nat@height>\textheight\textheight\else\Gin@nat@height\fi}
\makeatother
% Scale images if necessary, so that they will not overflow the page
% margins by default, and it is still possible to overwrite the defaults
% using explicit options in \includegraphics[width, height, ...]{}
\setkeys{Gin}{width=\maxwidth,height=\maxheight,keepaspectratio}
% Set default figure placement to htbp
\makeatletter
\def\fps@figure{htbp}
\makeatother

\usepackage{booktabs}
\usepackage{longtable}
\usepackage{array}
\usepackage{multirow}
\usepackage{wrapfig}
\usepackage{float}
\usepackage{colortbl}
\usepackage{pdflscape}
\usepackage{tabu}
\usepackage{threeparttable}
\usepackage{threeparttablex}
\usepackage[normalem]{ulem}
\usepackage{makecell}
\usepackage{xcolor}
\newcommand{\theHtable}{\thetable}
\makeatletter
\@ifpackageloaded{tcolorbox}{}{\usepackage[skins,breakable]{tcolorbox}}
\@ifpackageloaded{fontawesome5}{}{\usepackage{fontawesome5}}
\definecolor{quarto-callout-color}{HTML}{909090}
\definecolor{quarto-callout-note-color}{HTML}{0758E5}
\definecolor{quarto-callout-important-color}{HTML}{CC1914}
\definecolor{quarto-callout-warning-color}{HTML}{EB9113}
\definecolor{quarto-callout-tip-color}{HTML}{00A047}
\definecolor{quarto-callout-caution-color}{HTML}{FC5300}
\definecolor{quarto-callout-color-frame}{HTML}{acacac}
\definecolor{quarto-callout-note-color-frame}{HTML}{4582ec}
\definecolor{quarto-callout-important-color-frame}{HTML}{d9534f}
\definecolor{quarto-callout-warning-color-frame}{HTML}{f0ad4e}
\definecolor{quarto-callout-tip-color-frame}{HTML}{02b875}
\definecolor{quarto-callout-caution-color-frame}{HTML}{fd7e14}
\makeatother
\makeatletter
\@ifpackageloaded{caption}{}{\usepackage{caption}}
\AtBeginDocument{%
\ifdefined\contentsname
  \renewcommand*\contentsname{Table of contents}
\else
  \newcommand\contentsname{Table of contents}
\fi
\ifdefined\listfigurename
  \renewcommand*\listfigurename{List of Figures}
\else
  \newcommand\listfigurename{List of Figures}
\fi
\ifdefined\listtablename
  \renewcommand*\listtablename{List of Tables}
\else
  \newcommand\listtablename{List of Tables}
\fi
\ifdefined\figurename
  \renewcommand*\figurename{Figure}
\else
  \newcommand\figurename{Figure}
\fi
\ifdefined\tablename
  \renewcommand*\tablename{Table}
\else
  \newcommand\tablename{Table}
\fi
}
\@ifpackageloaded{float}{}{\usepackage{float}}
\floatstyle{ruled}
\@ifundefined{c@chapter}{\newfloat{codelisting}{h}{lop}}{\newfloat{codelisting}{h}{lop}[chapter]}
\floatname{codelisting}{Listing}
\newcommand*\listoflistings{\listof{codelisting}{List of Listings}}
\makeatother
\makeatletter
\makeatother
\makeatletter
\@ifpackageloaded{caption}{}{\usepackage{caption}}
\@ifpackageloaded{subcaption}{}{\usepackage{subcaption}}
\makeatother
\ifLuaTeX
  \usepackage{selnolig}  % disable illegal ligatures
\fi
\usepackage{bookmark}

\IfFileExists{xurl.sty}{\usepackage{xurl}}{} % add URL line breaks if available
\urlstyle{same} % disable monospaced font for URLs
\hypersetup{
  pdftitle={Class 08},
  pdfauthor={Sarah E. Grabinski},
  hidelinks,
  pdfcreator={LaTeX via pandoc}}

\title{Class 08}
\subtitle{DATA1220-55, Fall 2024}
\author{Sarah E. Grabinski}
\date{2024-09-16}

\begin{document}
\frame{\titlepage}

\begin{frame}{Homework 2}
\phantomsection\label{homework-2}
\begin{itemize}
\item
  Describing numerical distributions: modality, skew, outliers
\item
  Describing numerical distributions: appropriate summary statistics
\item
  Matching numerical distributions to their summary statistics, reading
  a boxplot
\item
  Calculating proportions from a contingency table
\end{itemize}
\end{frame}

\begin{frame}[fragile]{Homework 2}
\phantomsection\label{homework-2-1}
\begin{itemize}
\item
  \href{https://canvas.jcu.edu/files/3708401/download?download_frd=1}{Instructions}
  (\texttt{homework2\_instructions.pdf}), a
  \href{https://canvas.jcu.edu/files/3708307/download?download_frd=1}{Quarto
  markdown template} (\texttt{homework2\_template.qmd}), and an
  \href{https://canvas.jcu.edu/files/3708306/download?download_frd=1}{example
  HTML output} (\texttt{homework2\_example.html}) are available for
  download under Chapter 2 on the
  \href{https://canvas.jcu.edu/courses/36290/modules}{Modules} page in
  Canvas.
\item
  Upload \textbf{\emph{TWO}} (2) documents to
  \href{https://canvas.jcu.edu/courses/36290/assignments/451733}{Homework
  2} on the
  \href{https://canvas.jcu.edu/courses/36290/assignments}{Assignments}
  page in Canvas by \textbf{\emph{Friday 9/20/2024}} by
  \textbf{\emph{6:00pm}}: \texttt{homework2\_yourlastname.qmd} and
  \texttt{homework2\_yourlastname.html}
\item
  \href{https://canvas.jcu.edu/files/3708369/download?download_frd=1}{Video
  walk-through} of Homework 2 under Tutorials on the
  \href{https://canvas.jcu.edu/courses/36290/modules}{Modules} page in
  Canvas. Make sure you're caught up on the
  \href{https://canvas.jcu.edu/files/3695568/download?download_frd=1}{video
  walk-through of homework 1}.
\end{itemize}
\end{frame}

\begin{frame}{Late Policy}
\phantomsection\label{late-policy}
``This homework is due by 6:00pm on Friday, 9/20/24. No credit will be
lost for assignments received by 7:00pm to account for issues with
uploading. 10\% of the points will be deducted from assignments received
by 9:00am on Saturday, 9/21/24. Assignments turned in after this point
are only eligible for 50\% credit, so it benefits you to turn in
whatever you have completed by the due date.''
\end{frame}

\begin{frame}{How can I get help with homework?}
\phantomsection\label{how-can-i-get-help-with-homework}
\begin{itemize}
\item
  \textbf{\emph{Read the
  \href{https://canvas.jcu.edu/files/3669904/download?download_frd=1}{textbook}.}}
  Many of you are asking for additional examples. Luckily, there are
  tons we didn't go over in the textbook.
\item
  \textbf{\emph{Look at the
  \href{https://canvas.jcu.edu/courses/36290/assignments/451733}{homework}
  early}}. I can see in Canvas that many students didn't download the
  documents until 1-2 days before it was due. That's not a lot of time
  to get help.
\item
  \textbf{\emph{Ask a question on our
  \href{https://campuswire.com/c/G6427C531/feed}{Campuswire class
  feed}.}} I'm only one person, and I may not be able to give you a
  prompt answer. However, the 20+ other people in the class might be
  able to.
\item
  \textbf{\emph{Come to office hours.}} I will be available after class
  today (Monday 9/23/2024) and Wednesday 9/25/2024 from 2:30pm - 4:00pm.
  If you cannot make it, reach out to me to try and schedule an
  appointment.
\end{itemize}
\end{frame}

\begin{frame}{Last Time\ldots{}}
\phantomsection\label{last-time}
\begin{itemize}
\item
  Contingency tables: counts and proportions (frequencies)
\item
  Visualizing frequencies: bar plots, mosaic plots
\item
  Describing numerical relationships: linear vs nonlinear, strong vs
  weak
\item
  Visualizing 3+ variables
\end{itemize}
\end{frame}

\begin{frame}{Chapter 3 Objectives}
\phantomsection\label{chapter-3-objectives}
\begin{itemize}
\item
  Define probability, random processes, and the law of large numbers
\item
  Describe the sample space for disjoint and non-disjoint outcomes
\item
  Calculate probabilities using the General Addition and Multiplication
  Rules
\item
  Create a probability distribution for disjoint outcomes
\end{itemize}
\end{frame}

\begin{frame}{Defining Probability}
\phantomsection\label{defining-probability}
What does the word \textbf{\emph{probability}} mean to you?

\pause

\emph{``Highly likely''}

\pause

\emph{``Probably''}

\pause

\emph{``About even''}

\pause

\emph{``Almost no chance''}
\end{frame}

\begin{frame}{People interpret probability differently}
\phantomsection\label{people-interpret-probability-differently}
\begin{figure}[H]

{\centering \includegraphics{class08_files/mediabag/perceptions-of-proba.png}

}

\caption{Did your estimate fall within these ranges? Are these ranges
reasonable?}

\end{figure}%
\end{frame}

\begin{frame}{So what is probability?}
\phantomsection\label{so-what-is-probability}
\begin{tcolorbox}[enhanced jigsaw, colframe=quarto-callout-important-color-frame, coltitle=black, opacitybacktitle=0.6, toptitle=1mm, colbacktitle=quarto-callout-important-color!10!white, bottomrule=.15mm, left=2mm, leftrule=.75mm, rightrule=.15mm, title=\textcolor{quarto-callout-important-color}{\faExclamation}\hspace{0.5em}{Frequentist Definition}, breakable, opacityback=0, colback=white, arc=.35mm, titlerule=0mm, bottomtitle=1mm, toprule=.15mm]

The proportion of times that a particular outcome would occur if we
observed a random process an infinite number of times.

\end{tcolorbox}

\begin{itemize}
\item
  A \textbf{\emph{random process}} is one where you know which outcomes
  are possible (i.e.~the \textbf{\emph{sample space}}) but you don't
  know which outcome comes next

  \begin{itemize}
  \tightlist
  \item
    \emph{Examples: coin toss, die roll, stock market}
  \end{itemize}
\end{itemize}
\end{frame}



\end{document}
