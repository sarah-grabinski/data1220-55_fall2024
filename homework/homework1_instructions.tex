% Options for packages loaded elsewhere
\PassOptionsToPackage{unicode}{hyperref}
\PassOptionsToPackage{hyphens}{url}
\PassOptionsToPackage{dvipsnames,svgnames,x11names}{xcolor}
%
\documentclass[
  letterpaper,
  DIV=11,
  numbers=noendperiod]{scrartcl}

\usepackage{amsmath,amssymb}
\usepackage{iftex}
\ifPDFTeX
  \usepackage[T1]{fontenc}
  \usepackage[utf8]{inputenc}
  \usepackage{textcomp} % provide euro and other symbols
\else % if luatex or xetex
  \usepackage{unicode-math}
  \defaultfontfeatures{Scale=MatchLowercase}
  \defaultfontfeatures[\rmfamily]{Ligatures=TeX,Scale=1}
\fi
\usepackage{lmodern}
\ifPDFTeX\else  
    % xetex/luatex font selection
\fi
% Use upquote if available, for straight quotes in verbatim environments
\IfFileExists{upquote.sty}{\usepackage{upquote}}{}
\IfFileExists{microtype.sty}{% use microtype if available
  \usepackage[]{microtype}
  \UseMicrotypeSet[protrusion]{basicmath} % disable protrusion for tt fonts
}{}
\makeatletter
\@ifundefined{KOMAClassName}{% if non-KOMA class
  \IfFileExists{parskip.sty}{%
    \usepackage{parskip}
  }{% else
    \setlength{\parindent}{0pt}
    \setlength{\parskip}{6pt plus 2pt minus 1pt}}
}{% if KOMA class
  \KOMAoptions{parskip=half}}
\makeatother
\usepackage{xcolor}
\setlength{\emergencystretch}{3em} % prevent overfull lines
\setcounter{secnumdepth}{-\maxdimen} % remove section numbering
% Make \paragraph and \subparagraph free-standing
\ifx\paragraph\undefined\else
  \let\oldparagraph\paragraph
  \renewcommand{\paragraph}[1]{\oldparagraph{#1}\mbox{}}
\fi
\ifx\subparagraph\undefined\else
  \let\oldsubparagraph\subparagraph
  \renewcommand{\subparagraph}[1]{\oldsubparagraph{#1}\mbox{}}
\fi


\providecommand{\tightlist}{%
  \setlength{\itemsep}{0pt}\setlength{\parskip}{0pt}}\usepackage{longtable,booktabs,array}
\usepackage{calc} % for calculating minipage widths
% Correct order of tables after \paragraph or \subparagraph
\usepackage{etoolbox}
\makeatletter
\patchcmd\longtable{\par}{\if@noskipsec\mbox{}\fi\par}{}{}
\makeatother
% Allow footnotes in longtable head/foot
\IfFileExists{footnotehyper.sty}{\usepackage{footnotehyper}}{\usepackage{footnote}}
\makesavenoteenv{longtable}
\usepackage{graphicx}
\makeatletter
\def\maxwidth{\ifdim\Gin@nat@width>\linewidth\linewidth\else\Gin@nat@width\fi}
\def\maxheight{\ifdim\Gin@nat@height>\textheight\textheight\else\Gin@nat@height\fi}
\makeatother
% Scale images if necessary, so that they will not overflow the page
% margins by default, and it is still possible to overwrite the defaults
% using explicit options in \includegraphics[width, height, ...]{}
\setkeys{Gin}{width=\maxwidth,height=\maxheight,keepaspectratio}
% Set default figure placement to htbp
\makeatletter
\def\fps@figure{htbp}
\makeatother

\KOMAoption{captions}{tableheading}
\makeatletter
\@ifpackageloaded{caption}{}{\usepackage{caption}}
\AtBeginDocument{%
\ifdefined\contentsname
  \renewcommand*\contentsname{Table of contents}
\else
  \newcommand\contentsname{Table of contents}
\fi
\ifdefined\listfigurename
  \renewcommand*\listfigurename{List of Figures}
\else
  \newcommand\listfigurename{List of Figures}
\fi
\ifdefined\listtablename
  \renewcommand*\listtablename{List of Tables}
\else
  \newcommand\listtablename{List of Tables}
\fi
\ifdefined\figurename
  \renewcommand*\figurename{Figure}
\else
  \newcommand\figurename{Figure}
\fi
\ifdefined\tablename
  \renewcommand*\tablename{Table}
\else
  \newcommand\tablename{Table}
\fi
}
\@ifpackageloaded{float}{}{\usepackage{float}}
\floatstyle{ruled}
\@ifundefined{c@chapter}{\newfloat{codelisting}{h}{lop}}{\newfloat{codelisting}{h}{lop}[chapter]}
\floatname{codelisting}{Listing}
\newcommand*\listoflistings{\listof{codelisting}{List of Listings}}
\makeatother
\makeatletter
\makeatother
\makeatletter
\@ifpackageloaded{caption}{}{\usepackage{caption}}
\@ifpackageloaded{subcaption}{}{\usepackage{subcaption}}
\makeatother
\ifLuaTeX
  \usepackage{selnolig}  % disable illegal ligatures
\fi
\usepackage{bookmark}

\IfFileExists{xurl.sty}{\usepackage{xurl}}{} % add URL line breaks if available
\urlstyle{same} % disable monospaced font for URLs
\hypersetup{
  pdftitle={Homework 1},
  pdfauthor={DATA1220-55, Fall 2024},
  colorlinks=true,
  linkcolor={blue},
  filecolor={Maroon},
  citecolor={Blue},
  urlcolor={Blue},
  pdfcreator={LaTeX via pandoc}}

\title{Homework 1}
\usepackage{etoolbox}
\makeatletter
\providecommand{\subtitle}[1]{% add subtitle to \maketitle
  \apptocmd{\@title}{\par {\large #1 \par}}{}{}
}
\makeatother
\subtitle{Due: Friday 2024-09-06}
\author{DATA1220-55, Fall 2024}
\date{}

\begin{document}
\maketitle

Below are details on Homework 1, covering the introduction to R and
RStudio and Chapter 1 of
\href{https://www.openintro.org/book/os/}{OpenIntro Statistics}. This
homework is due by 6:00pm on Friday, 9/6/24. No credit will be lost for
assignments received by 7:00pm to account for issues with uploading.
10\% of the points will be deducted from assignments received by 9:00am
on Saturday, 9/7/24. Assignments turned in after this point are only
eligible for 50\% credit, so it benefits you to turn in whatever you
have completed by the due date.

A Quarto template has been posted on Canvas for you to use for this
assignment. Please use it. You can learn more about customizing Quarto
documents here:
\url{https://quarto.org/docs/get-started/hello/rstudio.html}

Please render your document as HTML and submit \textbf{\emph{BOTH}} the
.html file and .qmd file to the Homework 1 assignment on Canvas.

\subsection{Problem 1 - Survey}\label{problem-1---survey}

\subsubsection{Objectives}\label{objectives}

\begin{itemize}
\item
  Help Sarah get to know you and better understand the specific needs of
  the class
\item
  Register for Campuswire, where you will earn many of your
  participation points
\end{itemize}

\subsubsection{Survey}\label{survey}

A Google Forms survey was sent to your JCU email. I estimate it will
take 5-10 minutes to complete.

Points: 5

\subsubsection{Campuswire}\label{campuswire}

Instructions for registering to our class Campuswire forum were sent to
your JCU email. I have also posted our first discussion topic.
Interacting with it will earn you participation credit. Make sure to
check your notification settings so you don't miss anything you want to
interact with!

Points: 5

\emph{Note: I have accidentally set this up from my Case Western email.
I don't believe that will affect you all, but please let me know if
there are any issues.}

\subsection{Problem 2 - Interpreting
Studies}\label{problem-2---interpreting-studies}

\subsubsection{Objectives}\label{objectives-1}

\begin{itemize}
\item
  Identify populations
\item
  Identify sampling strategies
\item
  Describe the reliability, validity, and generalizability of different
  types of data
\end{itemize}

\subsubsection{The Studies}\label{the-studies}

Researchers in the UK wanted to answer the question of how much crime
there was in Britain and whether it was going up or down. They used 2
different approaches to gather data for their investigation, but they
need help determining the validity of their approach.

\subsubsection{Data Set 1}\label{data-set-1}

The Crime Survey for England and Wales is a survey in which
approximately 38,000 people are questioned about their experiences with
crime. People surveyed are 16 years of age or older and were not living
in communal residences. Answers are self-reported.

\subsubsection{Data Set 2}\label{data-set-2}

UK Police keep administrative records of crimes they have investigated.
Police use internal definitions of crimes and their discretion when
creating these records.

\subsubsection{Questions}\label{questions}

\begin{enumerate}
\def\labelenumi{\arabic{enumi}.}
\item
  In 1 sentence each, describe the study population of the data sets.
  (Points: 2)
\item
  In 1 sentence each, describe the sampling strategy of the data sets.
  (Points: 2)
\item
  In 1 sentence each, describe the sampled population of the data set.
  (Points: 2)
\item
  In 1 sentence, describe the target population of the study (Points: 1)
\item
  In a short paragraph (3-6 sentences), please describe\ldots{} (Points:
  3)

  \begin{enumerate}
  \def\labelenumii{\alph{enumii}.}
  \item
    the reliability of each data set
  \item
    the validity of each data set
  \item
    if conclusions based on each data set from the study population are
    generalizable to the target population
  \end{enumerate}
\end{enumerate}

\subsection{Problem 3 - Interpreting
Data}\label{problem-3---interpreting-data}

\subsubsection{Objectives}\label{objectives-2}

\begin{itemize}
\item
  Incorporate text and code into a polished document
\item
  Read in a .csv (Comma Separated Values) data set
\item
  Create a data dictionary
\item
  Use \texttt{ggplot2} to create a plot and interpret it
\end{itemize}

The template has been partially completed to help you with this portion
of the assignment. Please fill in the missing portions.

You may find helpful hints on the cheat sheets for the bonus portion
here: \url{https://ggplot2.tidyverse.org/}

\subsubsection{The Data}\label{the-data}

The Child Health and Development Studies investigate a range of topics.
One study, in particular, considered all pregnancies between 1960 and
1967 among women in the Kaiser Foundation Health Plan in the San
Francisco East Bay area.

\subsubsection{Questions}\label{questions-1}

\begin{enumerate}
\def\labelenumi{\arabic{enumi}.}
\tightlist
\item
  Read in the .csv document (Points: 1)
\item
  Print a summary of the data. (Points: 1)
\item
  Complete the data dictionary. (Points: 3)
\item
  Add the name of an explanatory (i.e.~independent) variable to the
  x-axis of the plot and a response (i.e.~dependent) variable to the
  y-axis of the plot. In 1 sentence, describe what you see. (Points: 3)
\item
  BONUS: Add features such as titles, axis labels, colors, shapes, etc.
  to enhance your data visualization (Points available: 2)
\item
  Render your document as an HTML file (Points: 2)
\end{enumerate}

\subsection{Reminder}\label{reminder}

My office hours are MW 2:30-4:00pm immediately after class and Friday by
appointment in Dolan E252. You may also post questions to crowdsource
help (and answer them for your classmates! and upvote them!) on our
Campuswire forum, earning additional participation credit in the
meantime.



\end{document}
